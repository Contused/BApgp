\chapter{Einleitung}
\label{c:einleitung}

%TODO


\section{Aufgabenstellung}
\label{s:aufgabenstellung}

Die Arbeit befasst sich mit den folgenden Aufgaben:
\begin{enumerate}
	\item \textbf{''Programmierung einer Browser-Extension zur Anzeige von Datenschutzinformationen im PlayStore''}
	\item \textbf{''Evaluierung von Caching Methoden einer Browser Extension''}
\end{enumerate}

Hauptaugenmerk ist die Erläuterung von Browser-Extensions, Umsetzung eines Beispiels und Limitationen.
Welche Arten von Speicher stehen einer Extension zur Verfügung und welche Performance-Ersparnisse kann durch Abspeichern von Daten die die Extension wiederholt benötigt eingespart werden. Welche Entlastung erfährt der Server mit Backend. Aufbau und Einbindung des ausgewählten Kandidaten


\section{Aufbau der Arbeit}
\label{s:aufbauderarbeit}

Zu Beginn werden Recherche Ergebnisse vorgestellt und ausgewertet.  Aus den dadurch gewonnenen Resultaten die Aufgaben genauer Definiert. Auf Basis der Recherche entsteht im 1. Teil eine Extension wobei der Fokus darauf liegt, dass diese möglichst übersichtlich bleibt und zur Evaluierung von Speichermethoden dient. Anschließend werden verschiedene Testläufe präsentiert bei denen bestimmte Methoden zur lokalen Speicherung von Daten unter den gleichen Rahmenbedingungen verwendet werden. Die Ergebnisse werden verglichen und den Erwartungen gegenübergestellt. Zuletzt wird ein Fazit gezogen.

















