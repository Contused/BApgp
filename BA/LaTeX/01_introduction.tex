\chapter{Einleitung}
\label{c:introduction}

A general introduction into the topic goes here. Here is the place to state problems, which should be solved with this thesis. Also background information and non-academic information can be writen here. This part should sensibilise the reader for the topic. Go from higher level into the specific topic of this thesis. Thus, give an embedding of this thesis into an overall context. State, why are you doing this (motivation) and what will be made better (reasons to conduct this research)


Kurz PGuard und Datenschutz erklären. Verbindung zur Extension und Abspeichern von Informationen.


\section{Ziel}
\label{s:purpose}

Die Arbeit befasst sich mit den folgenden Aufgaben:
\begin{enumerate}
	\item \textbf{''Programmierung einer Browser Extension zur Anzeige von Datenschutzinformationen im PlayStore''}
	\item \textbf{''Evaluierung von Caching Methoden einer Browser Extension''}
\end{enumerate}

State here, what the work will be trying to answer. But also, what this work is NOT about. Thus, state the scope of this work.

Hauptaugenmerk = Erläuterung von Browser-Extensions, Umsetzung eines Beispiels und Limitationen.
Welche Arten von Speicher stehen einer Extension zur Verfügung und welche Performance-Ersparnisse kann durch Abspeichern von Daten die die Extension wiederholt benötigt eingespart werden. Welche Entlastung erfährt der Server mit Backend. Aufbau und Einbindung des ausgewählten Kandidaten

Was kommt nicht vor? Anleitung zur Abspeicherung personenbezogener/privater Daten. Technischer Aufbau jedes Kandidaten.

\section{Structure}
\label{s:structure}

This Bachelor/Master Thesis is structured as follows. First, chapter~1 gives some background information about ... which will be used throughout this work. Sequentially, chapter~2 presents proposed solutions and results for the beforehand stated questions. The setup and results for the first question are line out in section~3. The results for the second problem are stated in section~4. Finally, the obtained results are discussed and summed up in chapter~5. This chapter also gives suggestions for future research.


Recherche => Extension theorie => Extension praxis => Speicher Theorie => Umsetzung => Evaluation

















